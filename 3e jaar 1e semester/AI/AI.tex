\documentclass[a4paper,12pt]{article}

\usepackage[dutch]{babel}
\usepackage{fancyhdr}
\usepackage{graphicx}
\usepackage[pdftex,bookmarks=true]{hyperref}
\usepackage[utf8]{inputenc}
\usepackage{fullpage}
\usepackage{parskip}
\usepackage{float}
\usepackage{subcaption}
\usepackage{listings}
\usepackage{lscape}

\title{Samenvatting AI \\ \large TIN 3 - HoGent}
\author{Lorenz Verschingel}

\begin{document}
\maketitle
\section{Inleiding}
\subsection{Wat is AI?}
AI kan men op 2 assen uitzetten.

\begin{enumerate}
\item gedrag of redenering
\item menselijk of rationeel
\end{enumerate}

Op deze manier komt men tot 4 grote categorieën.

\begin{enumerate}
\item Systemen die zich gedragen zoals mensen:

Om hier van AI te kunnen spreken moet het systeem voldoen aan de Turing test.

\item Systemen die denken zoals mensen
\item Systemen die rationeel denken
\item Systemen die zich rationeel gedragen

\subsection{Doel van AI}
\begin{itemize}
\item Beter begrijpen van mensen en dieren.
\item Denken en leren
\item Applicaties in computers
\end{itemize}
\end{enumerate}

\end{document}