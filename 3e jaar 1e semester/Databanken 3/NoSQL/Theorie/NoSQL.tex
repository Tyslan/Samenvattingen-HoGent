\documentclass[a4paper,12pt]{article}

\usepackage[dutch]{babel}
\usepackage{fancyhdr}
\usepackage{graphicx}
\usepackage[pdftex,bookmarks=true]{hyperref}
\usepackage[utf8]{inputenc}
\usepackage{fullpage}
\usepackage{parskip}
\usepackage{float}
\usepackage{subcaption}
\usepackage{listings}
\usepackage{lscape}

\title{Samenvatting Databanken III - NoSQL \\ \large TIN 3 - HoGent}
\author{Lorenz Verschingel}

\begin{document}
\maketitle
\section{Distributed Systems}
Een gedistribueerd systeem bestaat uit verschillende computers en software componenten.
Deze communiceren met elkaar via het netwerk en delen de resources.
Het systeem kan mainframes, workstations\ldots bevatten.

\subsection{Voordelen van het gedistribueerd systeem}
\begin{enumerate}
\item Reliability:

Als één machine crashed wordt de rest hierdoor niet beïnvloed.
\item Scalability:

Men kan makkelijk machines toevoegen als daar nood aan is.
\item Performance:

De verzameling aan processoren van de verschillende machines kan een hoger performantie bieden dan een centralized computer.
\item Open system:

Iedere service is toegankelijk voor iedere client.
\item Sharing of Resources
\item Flexibility
\item Speed
\end{enumerate}

\subsection{Nadelen van het gedistribueerd systeem}
\begin{enumerate}
\item Software:

Er is minder software support $\rightarrow$ Dit is het grootste probleem
\item Security:

Het delen van data tussen de machines verhoogt het risico op het vlak van security.
\item Troubleshooting
\item Networking
\end{enumerate}

\subsection{Scalability}
\subsubsection{Verticaal schalen}
Resources toevoegen aan een bestaande logische eenheid om de capaciteit te verhogen.
\subsubsection{Horizontaal schalen}
Meer machines (nodes) toevoegen aan het systeem om de capaciteit te verhogen.

\end{document}