\documentclass[a4paper,12pt]{article}

\usepackage[dutch]{babel}
\usepackage{fancyhdr}
\usepackage{graphicx}
\usepackage[pdftex,bookmarks=true]{hyperref}
\usepackage[utf8]{inputenc}
\usepackage{fullpage}
\usepackage{parskip}
\usepackage{float}
\usepackage{subcaption}
\usepackage{tabto}

\title{Samenvatting Onderzoekstechnieken \\ \large TIN 2 - HoGent}
\author{Lorenz Verschingel}

\begin{document}
\maketitle
\section{Het onderzoeksproces}
\subsection{De wetenschappelijke methode}
Aan de hand van \textbf{empirisch onderzoek} zijn we geïnteresseerd in volgende zaken:

\begin{enumerate}
\item Exploratie
\item Beschrijving
\item Voorspelling
\item Controle
\end{enumerate}

Het onderzoeksproces verloop normaal gezien als volgt:

\begin{enumerate}
\item Formuleren

Wat is de onderzoeksvraag
\item Exacte informatie behoefte definiëren

Welke specifieke vragen moeten we stellen
\item Uitvoeren onderzoek

Enquêtes, simulaties\dots
\item Verwerken gegevens

Statistische software
\item Analyseren gegevens

Uitvoeren statistische methodes
\item Conclusie schijven

Schrijven onderzoeksverslag
\end{enumerate}

\subsection{Basisconcepten in onderzoek}
\subsubsection{Variabelen en waarden}
Een variabele is een eigenschap van een object waardoor we objecten van elkaar kunnen onderscheiden.

Een waarde is een specifieke eigenschap, een invullen voor een variabele.

\subsubsection{Meetniveaus}
De kwalitatieve schalen zijn:

\NumTabs{6}
\begin{enumerate}
\item 	Nominaal:
		\tab{Categorieën}
		\tab{geslacht, ras, land\dots}
\item	Ordinaal:
		\tab{Volgorde}
		\tab{militaire rang, opleidingsniveau\dots}
\end{enumerate}

De kwantitatieve schalen zijn:
\NumTabs{6}
\begin{enumerate}
\item 	Interval:
		\tab{Meting: }
		\tab{nulpunt is onbelangrijk}
		\tab{graden Celsius}
\item	Ratio:
		\tab{Meting:}
		\tab{t.o.v. absoluut nulpunt}
		\tab{meter, Joule, kilogram}
\end{enumerate}

\subsubsection{Verbanden tussen variabelen}
Er is een verband tussen variabelen als hun waarde systematisch veranderen.

Men is vooral op zoek naar oorzakelijke verbanden:
\begin{itemize}
\item Frustratie leidt tot aggressie
\item Alcohol leidt tot minder oplettendheid
\end{itemize}
De oorzaak is hierbij de onafhankelijke variabele.

Het gevolg is de afhankelijke variabele.

Hierbij moet men wel opletten. Een verband tussen variabelen duidt niet noodzakelijk op een oorzakelijk verband.
\end{document}