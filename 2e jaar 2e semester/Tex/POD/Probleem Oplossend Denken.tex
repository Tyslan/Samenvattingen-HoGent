\documentclass[a4paper,12pt]{article}

\usepackage[dutch]{babel}
\usepackage{fancyhdr}
\usepackage{graphicx}
\usepackage[pdftex,bookmarks=true]{hyperref}
\usepackage[utf8]{inputenc}
\usepackage{fullpage}
\usepackage{parskip}
\usepackage{float}
\usepackage{subcaption}

\title{Samenvatting Probleem Oplossend Denken II \\ \large TIN 2 - HoGent}
\author{Lorenz Verschingel}

\begin{document}
\maketitle
\LARGE \textsc{Kansrekening}\normalsize
\section{Gebeurtenissen en hun kansen}
\subsection{Inleiding}
De kansrekening houdt zich bezig met de studie van gebeurtenissen of toevalsveranderlijken.

\subsection{Universum of uitkomstenruimte}
Het universum of de utkomstenruimte van een experiment is de verzameling van alle mogelijke uitkomsten van dit experiment en wordt genoteerd met $\Omega$.

Het is van belang dat de uitkomstenruimte volledig is: elke mogelijke
uitkomst van een experiment moet tot $\Omega$ behoren.
Bovendien moet elke uitkomst van een experiment overeenkomen met juist één element van $\Omega$.

\subsection{Gebeurtenis}
Een gebeurtenis is een deelverzameling van de uitkomstenruimte.
Een enkelvoudige of elementaire gebeurtenis is een \textit{singleton}.

Een \textit{samengestelde} gebeurtenis heeft cardinaliteit groter dan 1.

Gebeurtenissen die geen gemeenschappelijke uitkomsten hebben noemt men \textit{disjunct}.
Disjuncte gebeurtenissen kunnen dus nooit samen voorkomen.

\subsection{Kansen en kansruimte}
We wensen nu aan elke gebeurtenis A een getal te koppelen dat uitdrukt hoe waarschijnlijk het is dat deze gebeurtenis voorkomt bij het uitvoeren van een experiment.
We noemen dit getal de \textit{kans} of \textit{waarschijnlijkheid} van A, en we noteren deze kans als $P(A)$.

Het toekennen van kansen aan gebeurtenissen dient aan de volgende drie regels te voldoen:

\begin{enumerate}
\item Kansen zijn steeds positief: voor elke gebeurtenis $A$ geldt dat $P(A) \geq 0$.
\item De uitkomstenruimte heeft kans 1: $P(\Omega)=1$.
\item Wanneer $A$ en $B$ disjuncte gebeurtenissen zij dan is: $P(A\cup B)=P(A)+P(B)$.

Dit noemt men de \textbf{somregel}.
\end{enumerate}

Wanneer de functie $P$ aan de bovenstaande eigenschappen voldoet dan noemt ment het drietal $(\Omega, P(\Omega), P)$ een \textit{kansruimte}.

Kansen voldoen aan de volgende eigenschappen:
\begin{enumerate}
\item Voor elke gebeurtenis A geldt dat $P(\overline{A}) = 1-P(A)$

$1 = P(\Omega) = P(A\cup \overline{A}) = P(A) + P(\overline{A})$
\item De onmogelijke gebeurtenis heeft een kans nul: $P(\emptyset)=0$

$P(\emptyset)= P(\overline{\Omega}) = 1 - P(\Omega) = 0$
\item Als $A \subseteq B$ dan is $P(A)\leq P(B)$, dan geldt= $P(A) = P(B) - P(B\setminus A)$.
\item De \textbf{uitgebreide somregel} is: $P(A \cup B) = P(A) + P(B) - P(A \cap B)$

$P(A\cup B) = P(A \cup (B\setminus A))\\
= P(A) + P(B\setminus A)\\
= P(A) + P(B\setminus (A \cap B))\\
= P(A) +- P(B) - P(A \cap B)$
\end{enumerate}
\subsubsection{Eindig universum}
\end{document}